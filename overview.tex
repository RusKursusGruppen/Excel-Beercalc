% !TEX TS-program = pdflatex
% !TEX encoding = UTF-8 Unicode

% This is a simple template for a LaTeX document using the "article" class.
% See "book", "report", "letter" for other types of document.

\documentclass[11pt]{article} % use larger type; default would be 10pt

\usepackage[utf8]{inputenc} % set input encoding (not needed with XeLaTeX)

%%% Examples of Article customizations
% These packages are optional, depending whether you want the features they provide.
% See the LaTeX Companion or other references for full information.

%%% PAGE DIMENSIONS
\usepackage{geometry} % to change the page dimensions
\geometry{a4paper} % or letterpaper (US) or a5paper or....
% \geometry{margin=2in} % for example, change the margins to 2 inches all round
% \geometry{landscape} % set up the page for landscape
%   read geometry.pdf for detailed page layout information

\usepackage{graphicx} % support the \includegraphics command and options

% \usepackage[parfill]{parskip} % Activate to begin paragraphs with an empty line rather than an indent

%%% PACKAGES
\usepackage{booktabs} % for much better looking tables
\usepackage{array} % for better arrays (eg matrices) in maths
\usepackage{paralist} % very flexible & customisable lists (eg. enumerate/itemize, etc.)
\usepackage{verbatim} % adds environment for commenting out blocks of text & for better verbatim
\usepackage{subfig} % make it possible to include more than one captioned figure/table in a single float
% These packages are all incorporated in the memoir class to one degree or another...

%%% HEADERS & FOOTERS
\usepackage{fancyhdr} % This should be set AFTER setting up the page geometry
\pagestyle{fancy} % options: empty , plain , fancy
\renewcommand{\headrulewidth}{0pt} % customise the layout...
\lhead{}\chead{}\rhead{}
\lfoot{}\cfoot{\thepage}\rfoot{}

%%% SECTION TITLE APPEARANCE
\usepackage{sectsty}
\allsectionsfont{\sffamily\mdseries\upshape} % (See the fntguide.pdf for font help)
% (This matches ConTeXt defaults)

%%% ToC (table of contents) APPEARANCE
\usepackage[nottoc,notlof,notlot]{tocbibind} % Put the bibliography in the ToC
\usepackage[titles,subfigure]{tocloft} % Alter the style of the Table of Contents
\renewcommand{\cftsecfont}{\rmfamily\mdseries\upshape}
\renewcommand{\cftsecpagefont}{\rmfamily\mdseries\upshape} % No bold!

%%% END Article customizations

%%% The "real" document content comes below...

\title{Brief Article}
\author{The Author}
%\date{} % Activate to display a given date or no date (if empty),
         % otherwise the current date is printed 

\begin{document}

\section*{Generelt omkring ølregnskab}

\begin{itemize}
\item Arket burde regne de fleste ting ud selv - indtast kun indkøb, priser, streger, betaling etc.

\item Det er vigtigt at der hver dag tælles hvor meget der er tilbage af drikkevarer, specielt dem på streglisterne.

\item Det er nemmest hvis man laver afregning hver dag - så kan man se om russerne har sejlet i at sætte streger.

\item Man skal huske at forklare dem at prisen på øllene afhænger af hvor gode de er til at sætte streger.

\item Samtidig har russerne også mulighed for at holde styr med hvor mange penge de har brugt, og om de har taget nok med

\item Det er ikke nødvendigt at tænke i byttepenge, da arket også kan håndtere at vi skylder dem penge

\item Vejlederne afregner bare efter rusturen.

\item Man skal huske at bestille øl til tiden på DIKU - altså INDEN rusturen

\item Husk at tælle op ved ankomsten til hytten - tjek at ordresedlen passer med leveringen

\item Husk at tælle op inden afgang fra hytten - eller inden resten af øl afleveres tilbage

\item Husk at tælle hvor meget øl, der tages med i bussen på vej hjem

\end{itemize}

\section*{Resultatark:}

\begin{itemize}

\item Indkøbt: Penge, der er købt ind for

\item Betalt: Penge, der er kommet ind

\item Leveret tilbage: Beløb der er leveret tilbage for

\item Tilbage på DIKU: Beløb der er bragt varer tilbage til DIKU for

\item Manglende indbetaling: Beløb der mangles betalt

\item Overskud: Det overskud, der er kommet ud af ølsalget i år

\end{itemize}

\section*{Beholdning:}

\begin{itemize}

\item Stk pris tastes for hver indkøbt vare

\item Antal tastes for hver indkøbt vare

\item Sodavand: Bemærk at hver sodavandstype har forskellige indkøbspris - tastes ind hver for sig

\item Rødvin: Her tastes vin til mad ind - evt kan den bruges hvis russerne kan købe rødvin

\item Fun: Saftevand til mad på fad - husk at stykpris skal regnes ud som prisen for en halv liter

\item Indvielse: Ingredienser til indvielsen - både sprut og evt sodavand

\item Tour de Chambre: Hvis noget sådan laves tastes indkøb til dette ind

\item Div. spiritus: Inkluderer alt der ikke er plads til andet steds, som Dr. Nielsen fx.

\item Mad på fad: Det regnes ud hvad prisen er pr. glas - halvliters fadølsglas - ud fra prisen på vodka og Fun

\end{itemize}

\section*{Optælling}

\begin{itemize}

\item Start: Her udregnes (ud fra information tastet andetsteds) hvor stor en beholdning der er førstedag. Husk at tælle efter om tallene passer.

\item dag2-dagx: Der tælles op og indtastes den nuværende beholdning. Svind fra dagen før kan så ses under "Forsvundet".

\end{itemize}

\section*{Gratis}

\begin{itemize}

\item Oversigt over hvor meget sprut, der bliver regnet gratis med i ølregnskab - gratis som i russerne betaler ikke særskildt for dette.

\item RKG’s kasse: Det overskud der er aftalt at rusturen skal lave til RKG - beløbet kan justeres

\end{itemize}

\section*{Stregark - Onsdag m.m.}

\begin{itemize}

\item Rusnavne og vejledernavne tastes ind i onsdagsarket - i kolonne A - og portes derefter automatisk til følgede dage

\item Drikkevarer (øl mm.): Her tastes hvor mange streger den enkelte rus har sat ved de enkelte drikkevarer.

\item Immatrikulation: Forudbetaling af penge - normalt 100 kr, der skrives ud for hver - minus vejledere - if applicable

\item Tour de Chambre: I beholdningen står der en pris for hele arrangementet. Det deles med antallet af deltagere og prisen skrives ind i dagsarket - rund evt. op eller ned - if applicable

\item Rødvin: Jokerfelt til streglisten - navnet kan ændres i beholdningen og reflekteres derefter igennem arket

\item Til højre kan man se hvor meget de har drukket for den pågældende dag.

\item En kolonne til at taste ind hvad de betaler

\item En kolonne så man kan se hvormeget de stadig skylder (negative tal betyder de har penge til gode)

\item I bunden regnes ud hvad en øl skal koste for at det løber rundt i forhold til forsvunden mængde

\item Prisen sættes selv nedenfor - vejledernes kan sættes lavere end russernes

\item Nederst ses hvordan regnskabet løbet rundt indtil videre - og om det gratis er dækket ind endnu

\end{itemize}

\section*{Opsummering}

\begin{itemize}

\item Opsummering af forbrug: Det kan ses hvor meget hver enkelt har drukket indtil videre

\item Opsummering af betaling: Hvormange penge har hver enkelt brugt indtil videre

\end{itemize}

\end{document}
